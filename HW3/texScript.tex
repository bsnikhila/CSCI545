% --------------------------------------------------------------
% This is all preamble stuff that you don't have to worry about.
% Head down to where it says "Start here"
% --------------------------------------------------------------
 
\documentclass[12pt]{article}
 
\usepackage[margin=1in]{geometry} 
\usepackage{amsmath,amsthm,amssymb}
\usepackage[T1]{fontenc}
\usepackage{amsmath}
\usepackage{upgreek}
\usepackage{graphicx}
\graphicspath{ {images/Robotics_Assign3/} }
 
\newcommand{\N}{\mathbb{N}}
\newcommand{\Z}{\mathbb{Z}}
 
\newenvironment{theorem}[2][Theorem]{\begin{trivlist}
\item[\hskip \labelsep {\bfseries #1}\hskip \labelsep {\bfseries #2.}]}{\end{trivlist}}
\newenvironment{lemma}[2][Lemma]{\begin{trivlist}
\item[\hskip \labelsep {\bfseries #1}\hskip \labelsep {\bfseries #2.}]}{\end{trivlist}}
\newenvironment{exercise}[2][Exercise]{\begin{trivlist}
\item[\hskip \labelsep {\bfseries #1}\hskip \labelsep {\bfseries #2.}]}{\end{trivlist}}
\newenvironment{reflection}[2][Reflection]{\begin{trivlist}
\item[\hskip \labelsep {\bfseries #1}\hskip \labelsep {\bfseries #2.}]}{\end{trivlist}}
\newenvironment{proposition}[2][Proposition]{\begin{trivlist}
\item[\hskip \labelsep {\bfseries #1}\hskip \labelsep {\bfseries #2.}]}{\end{trivlist}}
\newenvironment{corollary}[2][Corollary]{\begin{trivlist}
\item[\hskip \labelsep {\bfseries #1}\hskip \labelsep {\bfseries #2.}]}{\end{trivlist}}
\newenvironment{problem}[2][Problem]{\begin{trivlist}
\item[\hskip \labelsep {\bfseries #1}\hskip \labelsep {\bfseries #2.}]}{\end{trivlist}}
 
\begin{document}

\topmargin=-0.45in
\evensidemargin=0in
\oddsidemargin=0in
\textwidth=6.5in
\textheight=9.0in
\headsep=0.25in 
 
% --------------------------------------------------------------
%                         Start here
% --------------------------------------------------------------
 
%\renewcommand{\qedsymbol}{\filledbox}
 
\title{CSCI 545: Homework 3}%replace X with the appropriate number
\author{Deepika Anand} %replace with your name
\maketitle
 
\begin{problem} 1 (a)
\end{problem}
\begin{Answer}
The coefficients are :-
\end{Answer}

\clearpage
\begin{problem} 1 (b)
\end{problem}
\begin{Answer}
Result wrt Filter function is\\
\includegraphics[width=18cm, height=20cm]{b_1}\\

Code:\\
\includegraphics[width=18cm, height=10cm]{b_2}\\

\textbf{Estimated delay}\\
In this case the estimated delay is about 5ms.
\end{Answer}

\clearpage
\begin{problem} 1 (c)
\end{problem}
\begin{Answer}
Result wrt Filter function is\\
\includegraphics[width=18cm, height=20cm]{c_1}\\

Code:\\
\includegraphics[width=18cm, height=10cm]{c_2}\\

\textbf{Values K and P wrt iterations}\\
\includegraphics[width=18cm, height=10cm]{c_3}\\

\pagebreak
\textbf{Initilization of P}\\
The P at iteration 0 was initialized as 1 and all subsequent values of P is computed using recursion formula.\\

\textbf{Estimated delay}\\
In this case is approx 3ms. \\

\textbf{Comparison Kalman vs Butterworth filtering}\\
Kalman performs better as delay is less and it converges faster. 
\end{Answer}



\end{document}

